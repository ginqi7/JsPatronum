% Created 2016-12-28 Wed 21:28
\documentclass[11pt]{article}
\usepackage[utf8]{inputenc}
\usepackage[T1]{fontenc}
\usepackage{fixltx2e}
\usepackage{graphicx}
\usepackage{longtable}
\usepackage{float}
\usepackage{wrapfig}
\usepackage{rotating}
\usepackage[normalem]{ulem}
\usepackage{amsmath}
\usepackage{textcomp}
\usepackage{marvosym}
\usepackage{wasysym}
\usepackage{amssymb}
\usepackage{hyperref}
\tolerance=1000
\date{\today}
\title{think}
\hypersetup{
  pdfkeywords={},
  pdfsubject={},
  pdfcreator={Emacs 25.1.1 (Org mode 8.2.10)}}
\begin{document}

\maketitle
\tableofcontents

\begin{enumerate}
\item javascript 中所有变量都可以归属于某一个对象的属性。
\end{enumerate}
\begin{verbatim}
var a = 1; // 在浏览器解释中,a 为 window 的一个属性,可以window.a 调用。
\end{verbatim}
\begin{enumerate}
\item javascript 对象可以用字典的形式调用属性
\end{enumerate}
\begin{verbatim}
person = new Object();
person.name = "gin";
person.age = 16;
console.log(person["name"]); // => "gin"
console.log(person["age"]); // => 16
\end{verbatim}

那么就可以通过ast找到所有的字面量,然后提出局部变量进行替换:
\begin{verbatim}
var a = document.getElementById('a');
a.innerHTML = 'hello world';
\end{verbatim}

\begin{verbatim}
(function(a, b, c, d, e, f){
    a[d] = a[b][c](d);
    a[d][e]=f;
})(this, 'document', 'getElementById', 'a', 'innerHTML', 'hello world');
\end{verbatim}

然后通过打乱字符串,用数组存储所有的字符,通过拼接数组来获得参数。
% Emacs 25.1.1 (Org mode 8.2.10)
\end{document}