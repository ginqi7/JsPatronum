% Created 2017-02-25 Sat 21:47
\documentclass[presentation]{beamer}
\usepackage[utf8]{inputenc}
\usepackage[T1]{fontenc}
\usepackage{fixltx2e}
\usepackage{graphicx}
\usepackage{longtable}
\usepackage{float}
\usepackage{wrapfig}
\usepackage{rotating}
\usepackage[normalem]{ulem}
\usepackage{amsmath}
\usepackage{textcomp}
\usepackage{marvosym}
\usepackage{wasysym}
\usepackage{amssymb}
\usepackage{hyperref}
\tolerance=1000
\usepackage{xeCJK}
\setCJKmainfont{Inziu Iosevka TC}
\usetheme{default}
\usecolortheme{}
\usefonttheme{}
\useinnertheme{}
\useoutertheme{}
\author{金琪琦}
\date{\today}
\title{JavaScript 代码混淆器}
\hypersetup{
  pdfkeywords={},
  pdfsubject={},
  pdfcreator={Emacs 25.1.1 (Org mode 8.2.10)}}
\begin{document}

\maketitle

\begin{frame}[label=sec-1]{背景}
\end{frame}
\begin{frame}[label=sec-2]{争议}
观点一:
\begin{quote}
前端代码公开,没有秘密,本身代码就没有保护的意义。
\end{quote}
观点二:
\begin{quote}
正因为前端程序是以源码的形式展示在用户面前,才需要通过代码混淆的方式使得代码难以阅读从而增强前端代码的安全性,进一步保护自己的系统。
\end{quote}
\end{frame}
\begin{frame}[label=sec-3]{折衷}
\end{frame}
\begin{frame}[label=sec-4]{开发环境}
\begin{itemize}
\item 语言:Java
\item 平台:跨平台
\end{itemize}
\end{frame}
\begin{frame}[label=sec-5]{混淆方法}
\begin{itemize}
\item 正则替换
\item 抽象语法树(AST)
\end{itemize}
\end{frame}
% Emacs 25.1.1 (Org mode 8.2.10)
\end{document}